\documentclass[a4paper,12pt]{amsart}
%\documentclass[a4paper,12pt]{jarticle}
\input definition.tex
\newtheorem{Exercise}{Problem}
%\newtheorem{Exercise}{Quiz}
\begin{document}
\begin{center}
{\large\bf Operator Analysis}
\end{center}

\bigskip

\noindent Period: 2010, winter term 

\noindent Lecturer:
Yamagami Shigeru

\medskip
\noindent Synopsis:\\
Graph theory is known to have vast applications in combinatorial
problems. Turning viewpoints into its analytical aspect,
we will be often faced with manipulating linear operators.
In this course, mainly working with finite graphs, 
the spectral analysis is performed for graphs having smaller operator norms,
through which we will get a good experience in mathematical classification problems.

% http://www-mi.sci.ibaraki.ac.jp/\~{}yamagami/set2005.pdf

\medskip
\noindent Prerequisite:\\
Set theory (basic notions and terminology)\\
Linear algebra (undergraduate level)\\
Analysis (limit arguments in euclidean spaces)

\medskip
\noindent Reference:\\ 
F.M.~Goodman, P.~de la Harpe and V.~Jones, 
Coxeter Graphs and Towers of Algebras, 
Springer, 1989.


\medskip
\noindent
Further Reading:\\
J.~Fr\"ohlich and T.~Kerler, 
Quantum Groups, Quantum Categories and Quantum Field Theory, 
Lec.~Notes in Math.~1542(1993), Springer.\\
M.~Reed and B.~Simon, Functional Analysis I, Academic Press, 1981. 
%http://sss.sci.ibaraki.ac.jp/teaching/oa2009.pdf

% \medskip
% \noindent Plan:

% \medskip
% \begin{tabular}{|l|l|}
% \hline
% 10/06 & Overviews on the course and the prerequisite\\
% \hline
% 10/13 & Spectral properties of hermitian operators\\
% \hline
% 10/20 & Problem session I\\
% \hline
% 10/27 & Graphs and isomorphisms\\
% \hline
% 11/10 & Norm of incidence matrices\\ 
% \hline
% 11/17 & Problem session II\\
% \hline
% 11/24 & Perron-Frobenius theorem\\
% \hline
% 12/1 & Spectral analysis of graphs of type A and D\\
% \hline
% 12/8 & Graphs of Norm $2$\\
% \hline
% 12/15 & Problem session III\\
% \hline
% 12/22 & Spectral analysis of graphs of type E\\
% \hline
% 01/12 & Computation of norms for graphs of type E\\
% \hline
% 01/19 & Classification of graphs of smaller norm\\
% \hline
% 01/26 & Problem session IV\\
% \hline
% \end{tabular}

% \quad
% \begin{minipage}{6cm}

% \end{minipage}
\vfill
\pagebreak
\section{Spectral Properties of Hermitian Matrices}

Given a square matrix $A = (a_{ij})$, let 
$\sigma(A)$ be the set of eigenvalues of $A$, which 
is referred to as the \textbf{spectrum} of $A$. 
There are several characterizations of $\sigma(A)$: 
(i) $\sigma(A) = \{ \lambda \in \C; \det( \lambda I -A) = 0\}$,  
(ii) $\sigma(A) = \{ \lambda \in \C; 
\text{$\lambda I -A$ is not invertible} \}$ and so on. 
The \textbf{spectral radius} of $A$ is then defined to be 
\[
r(A) = \max \{ |\lambda|; \lambda \in \sigma(A) \}.
\]

The \textbf{operator norm} of $A$ is, by definition, 
\[
\| A\| = \sup \{ \| A\xi\|; \xi \in \C^n, \| \xi\| = 1\}.
\]
Here $\| \xi\| = \sqrt{(\xi|\xi)}$ denotes 
the ordinary inner product norm 
of $\xi$. The norm $\| A\|$ is characterized as the 
largest constant satisfying 
\[
\| A\xi\| \leq \| A\|\, \| \xi\|
\quad
\text{for any $\xi$.}
\]
In other words, 
\[
\| A\| = \sup \{ \| A\xi\|/\| \xi\|; 0 \not= \xi \in \C^n\}.
\]

\begin{Example}
Let $A = (\delta_{ij} d_j)$ be a diagonal matrix. Then 
\[
\|
A\| = \max \{ |d_j|; j \geq 1\}. 
\]
\end{Example}

\begin{Exercise}
Show that 
\[
\| A\| = \sup \{ \| A \xi\|; \xi \in \C^n,  \| \xi\| \leq 1\}
= \sup \{ |(\xi|A\eta)|; \| \xi\| \leq 1, \| \eta\| \leq 1\}.
\]
\end{Exercise}

\begin{Exercise}
Compute the norm for diagonal matrices. 
\end{Exercise}

\begin{Proposition}
We have the inequality $r(A) \leq \| A\|$, i.e., 
\[
\sigma(A) \subset \{ \lambda \in \C; |\lambda| \leq \| A\| \}.
\]
\end{Proposition}

\begin{proof}
Let $\lambda \in \sigma(A)$ with $\xi$ an eigenvector. Then 
\[
|\lambda| \| \xi\| = \| A\xi\| \leq \| A\|\, \|\xi\|
\]
implies that $|\lambda| \leq \| A\|$. 
\end{proof}

\begin{Remark}
Here is a formula which refines the above estimate. 
(A proof can be found in any standard text on functional analysis.)
\[
r(A) = \lim_{n \to \infty} \| A^n\|^{1/n}. 
\]  
\end{Remark}

% As a slight generalization of matrix theory, we can work with 
% matrices indexed by a finite set $X$ as well: 
% a matrix is a family of numbers $A = (a_{xy})_{x, y \in X}$, 
% which operates on the inner product space $\ell^2(X)$ by 
% \[
% (A\xi)_x = \sum_{y \in X} a_{xy} \xi_y, 
% \quad 
% \xi = (\xi_x)_{x \in X} \in \ell^2(X).
% \]  

If we define the \textbf{hermitian conjugate} $A^*$ of a square matrix 
$A = (a_{jk})$ by $(A^*)_{jk} = \overline{a_{kj}}$, 
it is characterized by the relation 
\[
(A^*\xi | \eta) = (\xi| A\eta)\quad
\text{for $\xi, \eta \in \C^n$}.
\]

\begin{Proposition}
For any square matrix $A$, we have 
\[
\| A^* \| = \| A\|, 
\qquad 
\| A\|^2 = \| A^* A\|. 
\]
\end{Proposition}

\begin{Exercise}
Check these norm identities. 
\end{Exercise}

A matrix $A$ is \textbf{hermitian} if $A^* = A$. 
A matrix $U$ is \textbf{unitary} if $UU^* = U^*U = I$, where 
$I = (\delta_{j,k})$ denotes the unit matrix. 
($I$ stands for identity.)

A matrix $A$ is said to be \textbf{normal} if $AA^* = A^*A$. 
Normal matrices constitute a class which includes
hermitian and unitary ones.

\begin{Proposition}~ 
  \begin{enumerate}
  \item 
For a hermitian matrix $A$, $\sigma(A) \subset \R$.
\item 
For a unitary matrix $U$, 
$\sigma(U) \subset \{ z \in \C; |z| = 1\}$. 
  \end{enumerate}
\end{Proposition}

\begin{Exercise}
Determine the spectrum of a matrix $A$ 
which is hermitian and unitary at the same time.  
\end{Exercise}

\begin{Proposition}
Let $A$ be a normal matrix. 
\begin{enumerate}
\item
If $A\xi = \lambda \xi$ with $\xi \in \C^n$ and 
$\lambda \in \C$, then 
$A^*\xi = \overline{\lambda} \xi$.
\item 
If $A\xi = \lambda \xi$ and $A \eta = \mu \eta$ with 
$\lambda \not= \mu$, then $(\xi|\eta) = 0$.  
\end{enumerate}
\end{Proposition}

\begin{proof}
(i) This follows from 
\[
(A^*\xi - \overline{\lambda}\xi| A^*\xi - \overline{\lambda}\xi) 
= (A\xi - \lambda \xi | A\xi - \lambda \xi) = 0.
\]

(ii) is a consequence of 
\[
\lambda(\xi|\eta) = (A^*\xi|\eta) 
= (\xi|A\eta) = \mu (\xi|\eta).
\]
\end{proof}

\begin{Theorem}
Let $\{ \lambda_1, \dots, \lambda_n \}$ 
be the eigenvalue list (including multiplicity) of a normal matrix $A = (a_{ij})$. 
Then we can find a unitary matrix $U$ satisfying 
\[
A = U^* 
\begin{pmatrix}
\lambda_1 & 0 & 0\\
0 & \ddots & 0\\
0 & 0 & \lambda_n
\end{pmatrix}
U.
\]
\end{Theorem}

\begin{proof}
Let $\xi_1$ be a normalized eigenvector of $A$ of eigenvalue $\lambda_1$. 
By Gram-Schmidt's orthogonalization, we can find an orthonormal basis $(\xi_1,\dots,\xi_n)$ so that 
\[
(A\xi_1,\dots, A\xi_n) 
% = 
% (\xi_1,\dots, \xi_n) 
% \begin{pmatrix}
% \lambda_1 & * & \dots & *\\
% 0 & * & \dots & *\\
% \vdots & \vdots & \ddots & \vdots\\
% 0 & * & \dots & *
% \end{pmatrix}
= (\xi_1,\dots,\xi_n) 
\begin{pmatrix}
\lambda_1 & *\\
0 & B  
\end{pmatrix}
\]
with $B$ a square matrix of size $n-1$. By repeating the same procedure to $B$, we arrive at 
a unitary matrix 
$U$ such that $UAU^*$ is upper triangular. Since an upper triangular 
matrix is normal if and only if it is diagonal, we are done. 
\end{proof}

\begin{Corollary}
For a normal matrix $A$, we have $\| A\| = r(A)$.
\end{Corollary}

\begin{Exercise}
Investigate what we can say for the converse implication. 
\end{Exercise}

\begin{Exercise}
Check that an upper triangular matrix is normal if and only if 
it is diagonal. 
\end{Exercise}

Let $A$ be a hermitian matrix and set
\begin{align*}
\lambda_{max} &= \max \{ \lambda; \lambda \in \sigma(A) \},\\ 
\lambda_{min} &= \min \{ \lambda; \lambda \in \sigma(A) \}.
\end{align*}

\begin{Proposition}
We have the following expressions. 
\begin{align*}
\lambda_{max} &= \max \{ (\xi|A\xi); \| \xi\| = 1\},\\
\lambda_{min} &= \min \{ (\xi|A\xi); \| \xi\| = 1\}.
\end{align*}
\end{Proposition}

\begin{proof}
\[
\lambda_{min}(|\xi_1|^2 + \dots + |\xi_n|^2) 
\leq 
\lambda_1 |\xi_1|^2 + \dots + \lambda_n |\xi_n|^2 
\leq 
\lambda_{max}(|\xi_1|^2 + \dots + |\xi_n|^2).
\]
\end{proof}

\begin{Proposition}[Variational Principle]
Let $A$ be a hermitian matrix. If a unit vector $\xi$ attains 
the maximal value $\lambda_{max}$ of the function 
$(\xi| A\xi)$, then 
$A \xi = \lambda_{max} \xi$. 
\end{Proposition}

\begin{proof}
Use the quadratic inequality 
\[
(\xi + t\eta |(\lambda_{max}I - A)(\xi + t \eta)) \geq 0
\]
for any $t \in \R$, together with the assumption 
$(\xi|(\lambda_{max} I -A)\xi) = 0$. 
\end{proof}

\begin{Exercise}
Compute $\lambda_{max}$ and $\lambda_{min}$ for the 
real symmetric matrix 
\[
A = 
\begin{pmatrix}
a & b\\
b & c
\end{pmatrix}.
\]
\end{Exercise}

% Given a matrix $A \in M_{m,n}(\C)$, its (hermite) symmetrization 
% is the matrix $H \in M_{m+n}(\C)$ defined by 
% \[
% H = 
% \begin{pmatrix}
% 0 & A\\
% A^* & 0
% \end{pmatrix}.
% \]

% \begin{Lemma}
% \[
% \{ (\xi|H\xi); \| \xi\| = 1\} 
% = \{ \Re (\xi|A\eta); \| \xi\| \leq 1, \|\eta\| \leq 1\}.
% \]
% \end{Lemma}

% \begin{Corollary}
% \[
% \| H\| = \| A\|.
% \]
% \end{Corollary}

For a future reference, we recall permutation matrices here.
Given a permutation $\sigma \in S_n$ of degree $n$, 
the associated \textbf{permutation matrix}  $P_\sigma$ is defined by 
\[
(P_\sigma)_{ij} = \delta_{i,\sigma(j)} = 
\delta_{\sigma^{-1}(i),j}.
\]
The permutation matrix is orthogonal and satisfies 
$P_\sigma P_\tau = P_{\sigma\tau}$ 
for $\sigma, \tau \in S_n$. 

\begin{Exercise}
Check the multiplicativity property of permutation matrices. 
\end{Exercise}

\begin{Exercise}
Compute the spectrum of permutation matrices. 
\end{Exercise}

\section{Graphs and Adjacency Matrices}

%http://en.wikipedia.org/wiki/Seven_bridges_of_K%C3%B6nigsberg

The Euler's solution to the seven bridges problem of K\"onigsberg 
is known to be the birth of the notion of graph:
A connected unoriented graph allows an Euler trail  
if and only if the number of vertices of 
odd degree is less than three 
(the number of odd-degree vertices being always even 
for unoriented graphs). 

A \textbf{graph} is a quadruplet $(\Gamma,X, s, t)$, 
where $\Gamma$ and $X$ are sets with 
$s, t: \Gamma \to X$ maps. We call $\Gamma$ the set of 
edges, $X$ the set of vertices, whereas 
$s$ and $t$ are referred to as 
the source and target maps respectively. 
A graph is simply represented by the edge set when there arise 
no confusions. 

Given vertices $x$, $y \in X$, set 
\begin{gather*}
{}_x \Gamma = \{ \gamma \in \Gamma; t(\gamma) = x\}, 
\quad 
\Gamma_y = \{ \gamma \in \Gamma; s(\gamma) = y\},\\ 
{}_x \Gamma_y = 
\{ \gamma \in \Gamma; s(\gamma) = y, t(\gamma) = x \}.
\end{gather*}

A graph is \textbf{weakly finite} if ${}_x\Gamma_y$ is a finite set for 
any ordered pair of elements $(x, y)$ in $ X$, 
\textbf{locally finite} if both of ${}_x\Gamma$ and $\Gamma_x$ 
are finite sets for any $x \in X$, and 
\textbf{finite} if both of $\Gamma$ and $X$ are finite sets. 
Given a weakly finite graph $\Gamma$, 
its \textbf{adjacency matrix} is defined to be 
$\{ |{}_x\Gamma_y|\}_{x, y \in X}$. 

Two graphs $\Gamma$, $\Gamma'$ are said to be 
\textbf{isomorphic} if 
we can find bijections $\phi: \Gamma \to \Gamma'$ and 
$\phi^{(0)}: X \to X'$ satisfying 
\[
t(\phi(\gamma)) = \phi^{(0)}(t(\gamma)), 
\quad 
s(\phi(\gamma)) = \phi^{(0)}(s(\gamma))
\]
for any $\gamma \in \Gamma$. 

A graph $\Gamma$ is a \textbf{subgraph} of a graph $\Gamma'$ if 
$\Gamma \subset \Gamma'$, $X \subset X'$,
$s = s'|_{\Gamma}$ and $t = t'|_{\Gamma}$. 
By abuse of terminology, a graph isomorphic to 
a subgraph is also referred to as a subgraph. 

\begin{Example}
Cyclic permutations and circle graphs.
\end{Example}

By an involution of a graph $\Gamma$, we shall mean 
a bijection $\Gamma \ni \gamma \mapsto \gamma^{-1} \in \Gamma$ 
such that $t(\gamma^{-1}) = s(\gamma)$ and 
$(\gamma^{-1} )^{-1} = \gamma$. 
An \textbf{unoriented graph} is, by definition, a graph $\Gamma$ 
which is furnished 
with an involution satisfying 
$\gamma^{-1} = \gamma$ for $\gamma \in \Gamma$ satisfying $s(\gamma) = t(\gamma)$. 
% $\gamma^{-1} \not= \gamma$ for any $\gamma \in \Gamma$. 
% An unoriented graph is a $\Z_2$- set in an obvious way. 

% For an unoriented graph, we modify the definition of 
% adjacency matrix at diagonal components:
% $|{}_x\Gamma_x|/2$. 

Two square matrices are said to be \textbf{equivalent} if 
we can find a permutation matrix $P$ such that 
$P A P^{-1} = B$. 

\begin{Proposition}
There is a one-to-one correspondance between 
isopmorphism classes of finite graphs and 
equivalence classes of $\N$-valued square matrices. 

There is a one-to-one correspodance between 
isomorphism classes of finite unoriented graphs 
and equivalence classes of $\N$-valued symmetric matrices. 
\end{Proposition}

Let $G$ be a (at most) countable group with $1 \not\in S$ a set of generators.
Then the associated \textbf{Cayley graph} $\Gamma(G,S)$ is 
defined by $X = G$ and $\Gamma = \{ (g,ga); g \in G, a \in S \}$ 
with $t(g,ga) = g$ and $s(g,ga) = ga$. 

If the generator set $S$ further satisfies $S^{-1} = S$, i.e., $a \in S$ if and only if $a^{-1} \in S$, 
then $\Gamma(G,S)$ is unoriented with respect to the involution
$(g,ga)^{-1} = (ga,g)$. 

For the cyclic group $C_n$ with $a$ a generator, 
$\Gamma(C_n,\{ a\})$ is an oriented $n$-gon and 
$\Gamma(C_n, \{ a, a^{-1} \})$ an $n$-gon. 

\begin{Exercise}
Depict the oriented graph $\Gamma(\Z, \{ 1\})$ and the unoriented graph 
$\Gamma(\Z,\{ \pm 1\})$. 
\end{Exercise}

\begin{Exercise}
Depict the graph $\Gamma(F_2, \{ a^{\pm 1}, b^{\pm 1} \})$. Here $F_2$ denotes the free group generated by 
a two-element set $\{a, b\}$
\end{Exercise}

Given a finite graph $\Gamma$ with $A$ the incidence matrix, 
we define a linear operator $A: \ell^2(X) \to \ell^2(X)$ by 
\[
A \delta_x = \sum_{y \in X} A_{y,x} \delta_y. 
\]
By the \textbf{norm} $\| \Gamma\|$ of $\Gamma$, we shall mean 
the one for the linear operator $A$ and 
the \textbf{spectrum} $\sigma(\Gamma)$ of $\Gamma$ is, by definition, the spectrum of $A$. 

Here is a visual interpretation of eigenvector equations such as $A\xi = \lambda \xi$: 
Look at a vertex $x$ of the graph. Then 
\[
\lambda \xi(x) = \sum_{y} A_{x,y} \xi(y)
\]
with $A_{x,y}$ the number of edges from $y$ to $x$. 

\begin{Example}~ 
\begin{enumerate}
\item
The spectrum of cyclic permutations (oriented circle graphs). 
\[
\sigma(\Gamma) = \{ e^{2\pi ik/n}; 0 \leq k \leq n-1\}.
\]
\item
The spectrum of unoriented circle graphs. 
\[
\sigma(\Gamma) = \{ 2\cos(2\pi k/n); 0 \leq k \leq n-1\}.
\]
\item 
The spectrum and the norm of oriented lines (Jordan blocks).
\item 
The spectrum of unoriented lines.
\[
\sigma(\Gamma) = \{ 2\cos(\pi k/(n+1)); 1 \leq k \leq n\}.
\]
% \item 
% Oriented unilateral lines (unilateral shift and its adjoint). 
% \item 
% The unoriented unilateral line.
% \item 
% The oriented bilateral line.
% \item 
% The unoriented bilateral line. 
\end{enumerate}
\end{Example}

\begin{Exercise}
Check the case (i) and (iii).
\end{Exercise}

\begin{Exercise}
Compute the spectrum of the complete graph $K_n$ of $n$ vertices. 
\end{Exercise}

For the spectral analysis of $\Gamma(F_2,\{a^{\pm 1}, b^{\pm 1}\})$, 
we need some machinery of free probability theory. 

A \textbf{path} in a graph is a finite sequence 
$\gamma = \{ \gamma_1, \gamma_2, \dots, \gamma_n\}$ 
of edges satisfying 
\[
s(\gamma_j) = t(\gamma_{j+1}) 
\quad
\text{for $1 \leq j \leq n-1$.}
\]
The number $n$ is called the length of the path. 
We set $s(\gamma) = s(\gamma_n)$ and $t(\gamma) = t(\gamma_1)$. 

A graph is \textbf{connected} if, given two vertices $x$, $y$, 
we can find a path $\gamma$ such that $x = t(\gamma)$ and 
$y= s(\gamma)$.    

\begin{Remark}
Notice that a graph consisting of one vertex is connected 
if and only if the edge set is non-empty. 
\end{Remark}

\begin{Exercise}
Compute the spectrum and the norm for connected unoriented graphs 
of radial shape. 
\end{Exercise}

In what follows, we shall exclusively deal with unoriented graphs
and the adjective `unoriented' will be omitted 
unless otherwise stated. 



\section{Perron-Frobenius Theorems}
We shall describe basic results in Perron-Frobenius theory. 
To make the access easier, we restrict ourselves to the case of 
symmetric matrices. 

A symmetric matrix $A = (a_{ij})$ is said to be \textbf{non-negative} if 
$a_{ij}=a_{ji} \geq 0$ for all $i,j$.  
A non-negative symmetric matrix $A$ is said to be reducible if we can find 
a permutation matrix $P$ such that 
\[
PAP^{-1} = 
\begin{pmatrix}
* & 0\\
0 & *
\end{pmatrix}
\]
in a non-trivial manner, i.e., there is a finite non-empty 
proper subset $X$ of $\Gamma^{(0)}$ such that 
$(PAP^{-1})_{xy} = (PAP^{-1})_{yx} = 0$ for 
$x \in X$ and $y \not\in X$. 
Otherwise, we call it an \textbf{irreducible} matrix.

\begin{Proposition}
The following conditions for a symmetric matrix $A$ of non-negative 
entries are equivalent.
\begin{enumerate}
\item 
The matrix $A$ is irreducible.
\item 
For any $1 \leq i, j \leq n$, we can find an integer $k \geq 1$ such that 
$(A^k)_{ij} > 0$. 
\item 
Let $\Gamma$ be the graph associated to the symmetric matrix 
whose $(i,j)$-th component is set to be either $1$ or $0$ according to 
$A_{ij} > 0$ or $a_{ij} = 0$. Then $\Gamma$ is connected. 
\end{enumerate}
\end{Proposition}

\begin{Exercise}
Check the reducibility of the matrix
\[
\begin{pmatrix}
1 & 0 & 0 & 0 & 1 & 0\\
0 & 1 & 0 & 1 & 0 & 0\\
0 & 0 & 0 & 0 & 1 & 0\\
0 & 1 & 0 & 0 & 0 & 0\\
1 & 0 & 1 & 0 & 0 & 1\\
0 & 0 & 0 & 0 & 1 & 0
\end{pmatrix}.
\]
\end{Exercise}

\begin{Exercise}
Interpret the condition $(A^k)_{ij} > 0$ graphically.   
\end{Exercise}

\begin{Definition}
For a matrix $A$ of non-negative entries, an eigenvector 
of eigenvalue $r(A)$ (the spectral radius of $A$) 
with  non-negative components 
is called a \textbf{Perron eigenvector}. 
Note that $r(A) = \| A\|$ for a symmetric $A$. 
\end{Definition}

\begin{Theorem}
Any symmetric matrix $A$ of non-negative entries admits 
a Perron eigenvector: we can find a vector $\eta \not= 0$ satisfying 
$\eta_j \geq 0$ and $A\eta = \| A\| \eta$. 
\end{Theorem}

\begin{proof}
Since $A$ is hermitian, we can find a unit eigenvector 
$\xi$ of eigenvalue $\lambda$ such that $|\lambda| = \| A\|$. 
Let the unit vector $\eta = (\eta_j)$ 
be defined by $\eta_j = |\xi_j|$. Then 
\[
\| A\| = |\lambda| = |(\xi|A\xi)| \leq (\eta|A\eta) 
\leq \lambda_{max} \leq \| A\|
\]
implies $\lambda_{max} = \| A \|$ and 
$\lambda_{max} = (\eta|A\eta)$. By the variational principle, 
$\eta$ is a Perron eigenvector of $A$. 
\end{proof}

\begin{Theorem}
Let $A$ be an irreducible symmetric matrix of non-negative 
entries. 
\begin{enumerate}
\item 
Perron eigenvector is unique 
up to scalar multiplication and all of its components are 
strictly positive. 
\item 
Any eigenvector of maximal modulus eigenvalue is proportional to 
a Perron eigenvector. 
\item 
Any eigenvector of $A$ with non-negative components is a Perron 
eigenvector. 
  \end{enumerate}
\end{Theorem}

\begin{proof}
(i) Let $\eta = (\eta_1,\dots,\eta_n)$ be a Perron eigenvector 
and assume that $\eta_i>0$ for some $i$. Since $A$ is irreducible, 
we can find $N$ such that $(A^N)_{ji} > 0$ for any $j$ and then 
\[
\| A\|^N \eta_j = (A^N \eta)_j = \sum_k (A^N)_{jk} \eta_k 
\geq (A^N)_{ji} \eta_i > 0.
\]

(ii) If there exists an eigenvector which is not proportional 
to a given Perron eigenvector, we can find a eigenvector $\xi$ 
admitting a zero component. 
Then, by the proof of the previous theorem, $|\xi|$ is a Perron 
eigenvector, which contradicts with the strict positivity 
of components of Perron eigenvectors. 

(iii) Let $\xi$ be an eigenvector of 
$A$ with non-negative components. If its eigenvalue is different 
from $\| A\|$, then it is orthogonal to a Perron eigenvector. 
Again, by the strict positivity of Perron eigenvector, 
this is impossible. 
\end{proof}

\begin{Exercise}
Compute the Perron eigenvector of the matrix 
\[
\begin{pmatrix}
e^a & e^b\\
e^b & e^{-a}
\end{pmatrix}
\]
for $a, b \in \R$. 
\end{Exercise}

\begin{Exercise}
Investigate the validity of uniquness in the case of reducible matrices.
\end{Exercise}

The following simple observation is a key in the classification of 
connected graphs of smaller norm. 

\begin{Lemma}
Let $A$ and $B$ be symmetric matrices of non-negative entries 
such that $a_{ij} \leq b_{ij}$ for any $i,j$. Then 
$\| A\| \leq \| B \|$.

Furthermore, if $A$ is irreducible and $A \not= B$, 
the strict inequality $\| A \| < \| B\|$ holds. 
\end{Lemma}

\begin{proof}
Let $\xi$ be a normalized Perron eigenvector of $A$. 
Then we have 
\[
\| A \| = \| A\| (\xi|\xi) 
= (\xi|A\xi) \leq (\xi|B\xi) \leq \| B\| \|\xi\|^2 = \| B\|.
\]
If we assume $\| A \| = \| B\|$ in addition, 
then $(\xi|B\xi) = \| B\|$ and the variational principle 
implies that $\xi$ is a Perron eigenvector of $B$ as well. 

Now, from the irreducibility assumption of $B$, $\xi_j >0 $ 
for any $j$, which together with the relation 
$(B-A)\xi = \| B \| \xi - \| A \| \xi = 0$ shows that 
$B - A$ admits no positive entries, i.e., $B - A = 0$.
\end{proof}

\begin{Corollary}
Let $\Gamma'$ be a finite graph and 
$\Gamma$ be a connected proper subgrapah of $\Gamma'$. Then 
$\| \Gamma \| < \| \Gamma' \|$. 
\end{Corollary}

\begin{Remark}
Perron-Frobenius theorems are 
originally formulated for non-symmetric matrices, which correspond to 
adjacency matrices of oriented graphs. 
The norm-increasing principle then turns out to be 
applicable to oriented graphs as well. 
As another application, we remark here 
the asymptotic analysis of stochastic matrices and 
the google pagerank in evaluating websites. 
\end{Remark}


\section{Graphs of type A and D}
The graph $A_l$ ($l \geq 2$) is a linear graph of $l$ vertices. 
Number the vertices from one terminal sequencially and 
let $(c_1,c_2, \dots, c_l)$ be an eigenvector of eigenvalue 
$\lambda$. Then the eigenequation is of the form 
\begin{align*}
\lambda c_1 &= c_2,\\ 
\lambda c_j &= c_{j-1} + c_{j+1} 
\quad\text{for $2 \leq j \leq l-1$},\\
\lambda c_l &= c_{l-1}.
\end{align*}
If we introduce monic polynomials $\{ P_n(\lambda)\}_{n \geq 0}$ 
by 
\[
\lambda P_n(\lambda) = P_{n-1}(\lambda) + P_{n+1}(\lambda), 
\quad 
P_0(\lambda) = 1, P_1(\lambda) = \lambda, 
\]
then the eigenequation is reduced to the signle equation 
$c_{l+1} = c_1 P_l(\lambda) = 0$, i.e., 
$\det(\lambda I - A_l) = P_l(\lambda) = 0$. 
\[
P_2 = \lambda^2 -1, 
P_3 = \lambda^3 - 2\lambda, 
P_4 = \lambda^4 - 3\lambda^2 + 1, 
P_5 = \lambda^5 - 4\lambda^3 + 3\lambda.
\]
Returning to the original eigenequation, 
the generic part of the recursive relation is solved 
in terms of the characteristic roots $q, q^{-1}$ of the quadratic 
equation $t^2+1 = \lambda t$ satisfying 
$\lambda = q + q^{-1}$: 
$c_k = \alpha q^k + \beta q^{-k}$ ($1 \leq k \leq l$). 

If we take the initial condition $(q+q^{-1})c_1 = c_2$ into 
account, $\alpha + \beta = 0$ and we can set 
\[
c_k = \frac{q^k - q^{-k}}{q - q^{-1}} \equiv [k]_q
\]
up to multiplicative constants. Here $[k]_q$ is a Laurent polynomial of $q$ and 
can be evaluated at $q = \pm 1$: $[k]_{\pm 1} = \pm k$. 

Finally, the terminal matching $(q+q^{-1})c_l = c_{l-1}$ is 
satisfied if and only if $[l+1]_q = 0$. 
By the exchange symmetry between 
$q$ and $q^{-1}$, we may take 
$q = e^{\pi ki/(l+1)}$ ($1 \leq k \leq l$) 
with the associated eigenvalues given by 
$q + q^{-1} = 2\cos(k\pi/(l+1))$ ($1 \leq k \leq l$), i.e., 
\[
P_l(\lambda) = \prod_{k=1}^l 
\left( 
\lambda - 2 \cos \frac{k\pi}{l+1} \right). 
\]
The Perron eigenvector is obtained for the choice 
$q = e^{\pi i/(l+1)}$ with 
\[
\| A_l \| = q + q^{-1} = 2\cos \left( \frac{\pi}{l+1} \right), 
\quad 
c_k = \frac{\sin(k\pi/(l+1))}{\sin(\pi/(l+1))}.  
\] 

\hspace{2cm}
\input 
oa2006-1.tpc

\bigskip
\begin{Remark}~ 
\begin{enumerate}
\item
The expression $[n]_q$ is referred to as $q$-integer, which 
is known to be a source of many interesting identities (so-called $q$-analogues). 
Here $q$ is taken to be %can be considered to be 
a kind of deformation parameter with the ordinary integal relations 
recovered by taking the limit $q \to 1$. 
\item
The polynomial $P_n(\lambda)$ is also referred to as the Chebyshev 
polynomial of second kind: 
\[
% P_n(2\cos \theta) = \frac{\sin (n+1)\theta}{\sin \theta},
P_n(q+q^{-1}) = \frac{q^{n+1} - q^{-n-1}}{q - q^{-1}}
\quad \text{with} \ 
q = e^{i\theta}.
\]
\end{enumerate}
\end{Remark}

Consider the graph $D_l$ ($l \geq 4$), which is a graph 
with one edge and one vertex added to the graph $A_{l-1}$ in a 
minimal way. 
Let $(a,b,c[n]_q, c[n-1]_q, \dots, c[2]_q, c[1]_q)$ be 
an eigenvector of eigenvalue $q+q^{-1}$ ($l = n+2$). 
The choice of components 
automatically solves the most of eigenequation with 
the remaining part given by 
\[
(q+q^{-1}) a = (q+q^{-1}) b = c[n]_q, 
\quad 
(q+q^{-1}) c[n]_q = a + b + c[n-1]_q,
\]
which is solved by the equation 
\[
q^{2l-2} + 1 =0
\]
when $q + q^{-1} \not= 0$. 

After explicit computations, we find that it is reasonable to split into two cases: 

(i) The case of odd $l$: 
As solutions, we may take 
\[
q = ^{k\pi i/(2l-2)}, 
\quad 
k = 1, 3, 5, \dots, 2l-3. 
\]
Since $l$ is supposed to be odd, $[2]_q = q + q^{-1}  \not= 0$ for any of these and the eqigenequation 
is solved by
\[
a = b = \frac{[n]}{[2]}c. 
\]

If $q + q^{-1} = 0$, 
\[
[n] = \frac{q^{-l+1}}{q - q^{-1}} \left( (-1)^{l-2} - 1 \right) \not= 0
\]
is used to get 
\[
c=0, 
\quad a + b = 0. 
\]
Consequently, we see 
\[
\sigma(D_l) = \{ 0 \} \cup \{ 
2\cos\frac{k\pi}{2l-2}; k =1, 3, \dots, 2l-3 \}
\]
with all eigenvalues having multiplicity one. 

(ii) The case of even $l$: 

The solution 
\[
q = ^{k\pi i/(2l-2)}, 
\quad 
k = 1, 3, 5, \dots, 2l-3. 
\]
of $q^{2l-2} + 1 = 0$ satisfies $q + q^{-1} = 0$ exactly when $k = l-1$, whence each eigenvector is specified by 
\[
a = b = \frac{[n]}{[2]} c
\]
for $k \not= l-1$. 
 
If $q + q^{-1} = 0$, $[n] = 0$ but $[n-1] = (-1)^{l/2}$ shows that the eigenspace is specified by 
\[
a + b + (-1)^{l/2}c = 0, 
\]
i.e., zero is an eigenvalue of multiplicity $2$. 
Thus 
\[
\sigma(D_l) = \{ 
2\cos\frac{k\pi}{2l-2}; k =1, 3, \dots, 2l-3 \}, 
\]
which contains $0$. 

In either case, we have 
\[
\det( \lambda I - D_l) = \lambda \prod_{k=1}^{l-1} 
\Bigl(
\lambda - 2\cos \left( \frac{2k-1}{2l-2} \pi \right)  
\Bigr).
\]
The Perron eigenvector is obtained if we choose 
$q = e^{\pi i/2(l-1)}$, which particularly implies  
\[
\| D_l \| = 2\cos \left( \frac{\pi}{2l-2} \right), 
\]

\hspace{2cm}
\input oa2006-2.tpc

\noindent with $b = [l-2]/[2]$. 

\bigskip
As a final series of graphs, consider a graph $T_l$, 
which is the graph obtained from $A_l$ by adding one loop-edge 
at a terminal vertex. 
Let $([1]_q,[2]_q,\dots, [l]_q)$ be a Perron eigenvector. 
Then the matching condition arises at the loop-vertex: 
\[
(q+q^{-1})[l]_q = [l]_q + [l-1]_q.
\]
In terms of the expression 
\[
[n]_q = q^{n-1} + q^{n-3} + \dots + q^{-n+3} + q^{-n+1}, 
\]
we see that the above condition is equivalent to 
\[
0 = q^l - q^{l-1} + q^{l-2} + \dots + q^{-l} 
= q^{-l} \frac{1 - (-q)^{2l+1}}{1-(-q)},
\]
which has the solution (up to taking inverse in $q$) 
\[
\frac{1}{2l+1} \pi,\quad  
\frac{3}{2l+1} \pi,\quad \dots\ ,\quad  
\frac{2l-3}{2l+1} \pi,\quad 
\frac{2l-1}{2l+1} \pi,
\]
i.e., 
\[
\det(\lambda I - T_l) = \prod_{k=1}^l 
\Bigl(
\lambda - 2\cos \left( \frac{2k-1}{2l+1} \pi \right)  
\Bigr).
\]
As a result, we know that $q = e^{\pi i/(2l+1)}$ for 
the choice of Perron eigenvector and the graph norm is computed 
by 
\[
\| T_l\| = 2 \cos \left( \frac{\pi}{2l+1} \right).
\]

\hspace{2cm}
\input oa2006-3.tpc

\begin{Exercise}
Investigate the Perron eigenvalue of the following graph $L_n$. 
Show that 
$\lim_{n \to \infty} \| L_n\| = \frac{5}{2}$. 
\end{Exercise}

\hspace{2cm}
\input oa2006-18.tpc


\section{Graphs of Norm 2}

We shall describe most of the graphs of norm $2$. 
By adding one edge and one vertex to the graph of type A or D, 
there are series of graphs of norm $2$. 

\begin{Example}
Let ${\widetilde A}_l$ ($l \geq 2$) 
be the loop graph of $l+1$ vertices.

It is immediate to see that the Perron eigenvector is 
$(1,1,\dots, 1)$ with an eigenvalue of $2$.
\end{Example}

\hspace{3cm}
\input oa2006-4.tpc

\begin{Exercise}
Compute the eigenpolynomial $\det(\lambda I - {\widetilde A}_l)$; 
use the relation ${\widetilde A}_l = C + C^*$ with $C$ a cyclic 
permutation. 
\end{Exercise}

\begin{Example}
Let ${\widetilde D}_l$ ($l \geq 4$) be 
the I-shaped graph of $l+1$ vertices. 
Then 
the Perron eigenvector is 
$(1,1,1,1,2, 2, \dots,2)$ with an eigenvalue of $2$.
\end{Example}

\hspace{2cm}
\input oa2006-5.tpc

\bigskip
% \begin{Example}
% The graphs ${\widetilde E}_l$ ($l=6,7,8$) below are of norm $2$
% with the indicated Perron eigenvectors.
% \end{Example}

By adding of one or two loop-edges to the graph of type A or D, 
we obtain the following. 

\begin{Example}
The extended tadpole graph ${\widetilde T}_l$ ($l \geq 2$) of 
$l+1$ vertices. 
\end{Example}

\hspace{2cm}
\input oa2006-6.tpc

\begin{Example}
The double tadpole graph ${\widehat T}_l$ ($l \geq 1$) of 
$l$ vertices. 
\end{Example}

\hspace{2cm}
\input oa2006-7.tpc

\begin{Exercise}
Compute the Perron eigenvalue of the graph obtained 
by adding one free edge to each vertex of ${\widetilde A}_l$. 
\end{Exercise}

If we allow infinite graphs, 
there are three more graphs of norm $2$, 
which are limits of graphs of type A, D and T.

%Locally finite graphs and their graph norms. 

An infinite matrix $A = (a_{ij})$ is defined to be 
\textbf{locally finite} if 
\[
\{ i; a_{i,j} \not=0 \}
\quad
\text{and}
\quad 
\{ i; a_{j,i} \not= 0 \}
\]
are finite sets for any $j$. 
A vector of infinitely many components $\xi = \{ \xi_j\}$ is 
said to be essentially finite if we can find a finite subset $F$ 
such that $\xi_j = 0$ unless $j \in F$. 

An essentially finite vector is multiplied 
by a locally finite matrix, which is denoted by $A\xi$. 

A graph is locally finite if and only if the associated matrix is locally 
finite. 
The norm of a locally finite graph is defined exactly 
as in the case of finite graphs.  


\begin{Lemma}
For a locally finite matrix $A$, 
its norm is a limit of those for finite submatrices. 
More precisely, given a finite subset $F$ of indices, let 
$A_F$ be the finite matrix with index set restricted to $F$. 
Then we have 
\[
\| A\| = \lim_{F \to \Gamma^{(0)}} \| A_F\|.
\]
\end{Lemma}

\begin{Corollary}
For locally finite graphs, $\Gamma \subset \Gamma'$ implies 
$\| \Gamma\| \leq \| \Gamma'\|$. 
\end{Corollary}

\begin{Theorem}
Any connected locally finite infinite graph $\Gamma$ contains 
the graph $A_\infty$ as a subgraph and hence 
$\| \Gamma \| \geq 2$. 
The equality holds if and only if $\Gamma$ is one of 
$A_\infty$ (unilateral line), 
${\widetilde A}_\infty$ (bilateral line), 
$D_\infty$ or $T_\infty$. 
\end{Theorem}

Some discussions are in order for the first asserion. 
This can be seen as follows: Start with any point of the graph and try to delete edges so that 
it is maximal under the condition of connectedness. If such a cut is applied, we move to any point 
joined to the point just investigated and repeat the procedure. The inductive process like this 
spreads out to the whole points because of local finiteness assumption. 
After the total cutting, there remains a connected subgraph which loses connectivity 
if once any edge is removed, i.e., it is a tree subgraph $T$. 
Now start again with any point in $T$ and 
choose an edge connected to that point 
so that the connected part $T'$ beyond that edge is infinite. 
Repeat the argument to the tree $T'$ and the endpoint of the edge. 
 
For the proof of `only if' part, we need the result for T-shaped 
graphs, which will be checked in the next section. 

\hspace{2.3cm}
\input oa2006-8.tpc

\bigskip
\hspace{1cm}
\input oa2006-9.tpc

\bigskip
\hspace{1.5cm}
\input oa2006-10.tpc

\begin{Exercise}
Show that a connected locally finite infinite graph contains $A_\infty$ as a subgraph. 
Hint: By local finiteness the graph contains infinitely many vertices. Let $x_j$ be an infinite sequence of vertices. 
Choose a path for each pair $(x_j,x_{j+1})$ of vertices. Concatenate them and remove redundant edges and vertices. 
\end{Exercise} 

\section{Graphs of type E}
Let $T_{k,l,m}$ with $1 \leq k \leq l \leq m$ be 
the T-shaped graphs with lines of length 
$k$, $l$ and $m$ respectively ($T_{k,l,m}$ has $k+l+m$ edges). 
Since $T_{1,1,n} = D_{n+3}$, we focus on 
the other case. 

\hspace{2cm}
\input oa2006-11.tpc

\begin{Exercise}
Check the fact that $T_{k,l,m} \subset T_{k',l',m'}$ if and only 
if $k \leq k'$, $l \leq l'$ and $m \leq m'$. 
\end{Exercise}

We first seek for the graph of norm 2 among $T_{1,2,n-1}$'s. 
Let $(a,b_1,b_2,c_j)$ be a Perron vector. As in the case of type 
A and D, we can set $c_j = j$ ($1 \leq j \leq n$) for 
the longer line part. The eigenequation then takes the form 
\[
2a = n, \quad 
2b_1 = b_2, \quad 2b_2 = b_1 + n, \quad 
2n = a + b_2 + n-1, 
\]
which has the solution
\[
n =6, a = 3, b_1 = 2, b_2 = 4.
\]

Next we see for the graph of norm 2 among $T_{1,3,n-1}$. 
For the Perron eigenvector $(a, b, 2b, 3b, c_j=j)$, we have 
the equation 
\[
2a = n, \quad 
6b = 2b + n, \quad 
2n = a + 3b + n-1
\]
with the solution 
\[
n = 4, a = 2, b = 1.
\]

Similarly, the Perron eigenvector 
$(a, 2a, b, 2b, c_j = j)$ for $T_{2,2,n-1}$ admits 
an eigenvalue $2$ if and only if 
\[
n=3, a = b = 1.
\]

It is conventional to denote these graphs by 
${\widetilde E}_6 = T_{2,2,2}$, ${\widetilde E}_7 = T_{1,3,3}$
and ${\widetilde E}_8 = T_{1,2,5}$. 

\hspace{3cm}
\input oa2006-12.tpc

\bigskip
\hspace{2cm}
\input oa2006-13.tpc

\bigskip
\hspace{1.5cm}
\input oa2006-14.tpc

\begin{Theorem}
The graph $T_{k,l,m}$ has a norm $\leq 2$ if and only if 
$(k,l,m)$ is in the following list.
\begin{enumerate}
\item 
\[
(k,l,m) = (1,2,5), (1,3,3), (2,2,2).
\]
\item 
\[
(k,l,m) = (1,1,m\geq 1), (1,2,2), (1,2,3), (1,2,4). 
\]
\end{enumerate}
Write 
% $D_l = T_{1,1,l-3}$ ($l \geq 4$), 
$E_6 = T_{1,2,2}$, $E_7 = T_{1,2,3}$, $E_8 = T_{1,2,4}$.
\end{Theorem}

\begin{proof}
The list in (i) has a norm of 2 as observed already and hence 
$T_{k,l,m}$ with $(k,l,m)> (2,2,2)$ 
or $(k,l,m) > (1,3,3)$ or 
$(k,l,m) > (1,2,5)$ (in the inclusion ordering)
has a norm $> 2$. 
The remaining is in the list (ii). The graph $E_l$ ($l=6,7,8$) 
has a norm smaller than $2$ as a subgraph of 
${\widetilde E}_l$. The graph $D_l$ has a norm of 
$2 \cos (\pi/(2l-2))$. 
\end{proof}

Consider $T_{1,2,n-1}$ and let 
$(a, b_1, b_2, c_j)$ be a Perron eigenvector with 
$c_j = (q^j - q^{-j})/(q-q^{-1})$ for $1 \leq j \leq n$ and 
an eigenvalue $q + q^{-1}$. 
This choice solves the eigenequation for the longest line. 
The remaining equations are 
\begin{align*}
(q+q^{-1}) a' &= q^n - q^{-n},\\
(q+q^{-1}) b_1' &= b_2',\\
(q+q^{-1}) b_2' &= b_1' + q^n - q^{-n},\\
(q+q^{-1}) (q^n - q^{-n}) &= a' + b_2' + q^{n-1} - q^{-n+1}
\end{align*}
with $a' = (q - q^{-1})a$ and $b_j' = (q - q^{-1})b_j$. 
Under the condition that $(q-q^{-1})(q+q^{-1})(q^2 + 1 + q^{-2}) \not= 0$, 
%$1 < q+q^{-1} < 2$, 
these are equivalent to requiring
\[
(q^5 - q^{-5}) (q^n - q^{-n}) 
= (q+q^{-1})(q^3-q^{-3}) (q^{n-1} - q^{-n+1})
\]
together with 
\[
a' = \frac{q^n-q^{-n}}{q+q^{-1}}, 
\quad
b_1' = \frac{q^n-q^{-n}}{q^2+q^{-2}+1}, 
\quad
b_2' = (q+q^{-1}) b_1.
\]

We shall write down explicitly for $n=3,4,5$ 
(note $q-q^{-1} \not= 0$). 
%for $n=4$, expand 
%$(q^4 + q^2 + 1 + q^{-2} + q^{-4})(q^2+q^{-2}) = (q^2 + 1 + q^{-2})^2$. 
%for $n=5$, divide by $(q^2-1)^2$ after expanding
%(q^{10}-1)(q^{2n} - 1) = q^2(q^2+1)(q^6-1)(q^{2n-2} - 1)@

$n=3$:
\[
q^8 - q^4 + 1 = \frac{q^{12} +1}{q^4+1} = 0.
\]

$n=4$: 
\[
q^{12} - q^6 + 1 = \frac{q^{18}+1}{q^6+1} = 0.
\]

$n=5$: 
\[
q^{16} + q^{14} - q^{10} - q^8 - q^6 + q^2 + 1 = 0.
\]

For $n=3$ or $n=4$, we see that 
the choice $q= e^{i\pi/12}$ or $q = e^{i \pi/18}$ gives 
positive components of Perron eigenvector, which particularly 
shows $\| E_6 \| = 2 \cos(\pi/12)$ and 
$\| E_7 \| = 2 \cos(\pi/18)$. 

For the case $n=5$, we may expect a similar conclusion: 
the last equation is a cyclotomic polynomial of $q^2$. 
Since 
\[
\phi(2^a 3^b 5^c 7^d \cdots) = 
\phi(2^a) \phi(3^b) \phi(5^c) \phi(7^d) \dots 
= 2^{a-1} \cdot 2 3^{b-1} \cdot 4 5^{c-1} \cdot 6 7^{d-1} \cdots, 
\]
the cyclotomic polynomial of degree $8$ is for $2^4 = 16$, 
$2^3 3 = 24$ or $2\cdot 3\cdot 5 = 30$. 
By an explicit computation 
based on the exclusion-inclusion principle, 
\begin{align*}
\Phi_{16}(t) &= \frac{t^{16} -1}{t^8 - 1} = t^8 + 1,\\
\Phi_{24}(t) &= \frac{(t^{24}-1)(t^4-1)}{(t^{12}-1)(t^8-1)} 
= t^8 - t^4 + 1,\\
\Phi_{30}(t) &= \frac{(t^{30}-1) (t^5-1) (t^3-1)(t^2-1)} 
{(t^{15}-1) (t^{10}-1) (t^6-1) (t-1)} 
= t^8 + t^7 - t^5 - t^4 - t^3 + t + 1.
\end{align*}
Comparing this with the equation for $q$, we conclude that 
$q^2$ is a $30$-th primitive root of unity. 
In fact, for the choice $q = e^{i\pi/30}$, all the components 
of the relevant eigenvector is positive. 
Thus $\| E_8\| = 2 \cos (\pi/30)$. 

\bigskip
\hspace{3cm}
\input oa2006-15.tpc

\bigskip
\hspace{3cm}
\input oa2006-16.tpc


\bigskip
\hspace{3cm}
\input oa2006-17.tpc

\begin{Exercise}~ 
  \begin{enumerate}
  \item 
For integers $m$, $n$ satisfying $1 \leq m \leq n$, show the identity 
\[
[m][n] = [n+m-1] + [n+m-3] + \dots + [n-m+3] + [n-m+1]. 
\]
\item 
Derive the identity 
\[
[5][n] - [2][3][n-1] = [n+4] - [n] - [n-2]
\]
and relate this with with the eqigenequation for $T_{1,2,n-1}$. 
  \end{enumerate}
\end{Exercise}

\begin{Exercise}
Investigate the Perron eigenvalue of the graph $T_{n,n,n}$ and 
show that 
\[
\lim_{n \to \infty} \| T_{n,n,n} \| = \frac{3}{\sqrt{2}}.
\]
\end{Exercise}

\begin{Exercise}
Show that $\| T_{k,l,m} \| < \frac{3}{\sqrt{2}}$. 
\end{Exercise}


\section{Classification of Connected Graphs of Smaller Norm}
\begin{Theorem}
Finite connected graphs of norm $2$ are 
exactly one of the followings.
\begin{enumerate}
\item 
${\widetilde A}_l$ ($l \geq 1$).
\item 
${\widetilde D}_l$ ($l \geq 3$). 
\item 
${\widetilde T}_l$ ($l \geq 2$).
\item
${\widehat T}_l$ ($l \geq 1$). 
\item 
${\widetilde E}_l$ ($l=6,7,8$). 
\end{enumerate}
\end{Theorem}

\begin{Theorem}
Any connected graph of norm smaller than $2$ is contained 
in a connected graph of norm $2$ and isomorphic to 
one of the following graphs. 
\begin{enumerate}
\item 
$A_l$ ($l \geq 2$).
\item 
$D_l$ ($l \geq 3$).
\item 
$T_l$ ($l \geq 1$). 
\item 
$E_l$ ($l=6,7,8$). 
\end{enumerate}
\end{Theorem}

Let $\| \Gamma \| \leq 2$. 
Since $\| {\widehat T}_l \| = 2$ for $l \geq 1$, 
if $\Gamma$ contains two or more loop edges, 
it must be ${\widehat T}_l$. 
Since $\| {\widetilde T}_l \| = 2$ for $2 \geq 1$, 
if $\Gamma$ contains a triple vertex and a loop edge 
together, it should be ${\widetilde T}_l$. 

So assume now that $\Gamma$ contains no loop edges. 
Since $\| {\widetilde A}_l \| = 2$, $\Gamma$ should not contain 
a proper closed circuit; $\Gamma$ is ${\widetilde A}_l$ or a tree. 
Since $\| {\widetilde D}_4 \| = 2$, 
$\Gamma$ does not contain a quadruple vertice properly; 
$\Gamma = {\widetilde D}_4$ or $\Gamma$ contains no quadruple 
vertices. 

Since $\| {\widetilde D}_l \| =2$ for $l \geq 5$, 
$\Gamma = {\widetilde D}_l$ if $\Gamma$ contains two or more triple points and $\Gamma = T_{k,l,m}$ 
if $\Gamma$ contains one triple vertex. 
If there is no triple vertex, $\Gamma = A_l$ has norm less 
than $2$. 

From $\| T_{2,2,2} \| = 2$, we see $\| T_{2,l,m} \| > 2$ 
for $(2,l,m) > (2,2,2)$ and are reduced to the case 
$T_{1,l,m}$. Again $\| T_{1,3,3} \| = 2$ shows that 
$\| T_{1,l,m} \| > 2$ for $(1,k,m) > (1,3,3)$. 

Since $\| T_{1,2,5} \| = 2$, we see 
$\| T_{1,2,m} \| > 2$ for $m \geq 6$. 

Finally $\| T_{1,1,m} \| = 2\cos (\pi/2(m+2)) < 2$ for 
$m \geq 1$. 

% \begin{Exercise}
% Supply the proof Theorem~5.7. 
% \end{Exercise}

\begin{Exercise}
Investigate the norm of $T_{1,n,n}$ and show that 
$\lim_{n \to \infty} \| T_{1,n,n} \| = \sqrt{2+\sqrt{5}}$. 
Note that $\sqrt{2 + \sqrt{5}} = \phi^{1/2} + \phi^{-1/2}$ with 
$\phi = (1+\sqrt{5})/2$ (the Golden ratio).   
\end{Exercise}

\begin{Exercise}
Let $C_n$ be the graph obtained from ${\widetilde A}_n$ 
by adding one more edge and vertex. 
% Investigate the norm of $C_n$ and s
Show that 
$\|C_n\| > \|C_{n+1}\|$ for $n \geq 1$ (decreasing!) and 
$\lim_{n \to \infty} \| C_n \| = \sqrt{2 + \sqrt{5}}$. 
\end{Exercise}

\begin{Exercise}[Challenging]
Try to extend the classification results to oriented graphs.
\end{Exercise}

\section{Fusion Rule Algebras}

\begin{Definition}
A *-algebra $\C[S] = \sum_{s \in S} \C s$ with a distinguished 
countable basis $S$ containing the unit element $1$ is called 
an {\bf fusion rule algebra} (or simply fusion algebra)  
if the following two conditions are satisfied.
\begin{enumerate}
\item (Positivity) 
For $x$, $y$, $z \in S$, the coefficient $N_{xyz}$ of $1$ in $xyz$
(i.e., $xyz = N_{xyz}1 + \dots$) is a non-negative integer.
\item (Duality) 
The $*$-operation makes $S$ invariant globally
(i.e., $x \in S$ implies $x^* \in S$) and satisfies 
$N_{x^*y} = \delta_{x,y}$.
\end{enumerate}
\end{Definition}


\begin{Lemma}~
Let $\C[S]$ be a fusion algebra.
\begin{enumerate}
\item
If $N$ is extended to $\C[S]$ by
$N(\sum_{s \in S} c(s)s) = c(1)$, 
then it is a tracial state in the sense
that $N(ab) = N(ba)$, $N(a^*) = \overline{N(a)}$ 
for $a$, $b \in \C S$ and $N(1) = 1$.
Moreover, we have
\[
N(c^*c) = \sum_{s \in S} |c(s)|^2.
\]
\item
For $x$, $y \in S$,
\[
xy = \sum_{s \in S} N_{xys^*}s.
\]
\item
For $x$, $y$, $z \in S$, we have 
\[
N_{xyz} = N_{yzx},\quad N_{xyz} = N_{z^*y^*x^*}.
\]
\end{enumerate}
In view of the second statement, we occasionally use the notation
$N_{xy}^z$ of structure constants for $N_{xyz^*} = N_{z^*xy}$.
\end{Lemma}

\begin{proof}
(i) $N(ab) = N(ba)$ as well as the formula for $N(c^*c)$ is 
a consequence of the duality $N_{x^*y} = \delta_{x,y}$ and 
the relation $N(a^*) = \overline{N(a)}$ follows from 
$N(x) = \delta_{x,1}$
for $x \in S$.

(ii) If $xy = \sum_s c(s)s$ with $c(s) \in \C$, then the application
of the tracial state $N$ to $xyz^*$ yields
\[
N_{xyz^*} = \sum_{s \in S} c(s) N_{sz^*}
= \sum_{s \in S} c(s) \delta_{z,s}
= c(z).
\]

(iii) is immediate from the trace property of $N$.
\end{proof}

\begin{Definition}
Consider a *-algebra $\C[S]$ with a distinguished linear basis 
$S$ containing the unit element $1$ such that 
(i) the structure constants are non-negative reals, 
(ii) the set $S$ is globally invariant under the *-operation and 
(iii) the coefficient of $1$ in the expansion of $s^*s$ is 
strictly positive for any $s \in S$. 
From the previous lemma, fusion algebras fall into this class. 
\end{Definition}

\begin{Definition}
A multiplicative linear functional $d$ on a *-algebra $\C[S]$ 
in the above definition 
is called a dimension function if $d(s) > 0$ and 
$d(s^*) = d(s)$ for any $s \in S$.
\end{Definition}

\begin{Proposition}[Sunder]
Dimension function exists and is unique 
for a finite-dimensional *-algebra $\C[S]$. 
\end{Proposition}

\begin{proof}
Let $C = \sum_{s \in S} s \in \C[S]$ and consider 
the linear operator on $\C[S]$ defined by the right multiplication 
of $C$. Then it is irreducible as a positive matrix (operator) 
by the property $s*s = c 1 + \dots$ with $c>0$. 
Let $\xi = \sum_{x \in S} \xi(x) x$ be its 
Perron-Frobenius eigenvector: $\xi(x) > 0$ for $x \in S$ and 
$\xi C$ is a positive multiple of $\xi$. 

Since $s\xi \in \C[S]$ is again an eigenvector with 
non-negative coefficients, 
the uniqueness of Perron-Frobenius vector implies that 
$s\xi$ is proportional to $\xi$: let $d(s)>0$ be defined by 
$s\xi = d(s)\xi$. Clearly this defines a multiplicative 
functional on $\C[S]$. 

Let $d'$ be another multiplicative functional satisfying 
$d'(s)>0$ for $s \in S$. Then $\sum_{s \in S} d'(S) s$ 
is 
\end{proof}
\end{document}


\section{Operator Norm}
We shall here review some of basic notions in functional 
analysis.  
Recall that a (positive definite) inner product space is 
a complex vector space $\cH$ 
with a specified positive inner product. 
The numerical vector space $\C^n$ is furnished with 
the standard inner product
\[
(\xi|\eta) = \sum_{j=1}^n \overline{\xi_j} \eta_j.
\]
More generally, the free vector space 
$\C S$ generated by a set $S$ is an inner product space 
so that $S$ is an orthonormal basis. 

Any inner product space admits a natural norm: 
$\| \xi\| = \sqrt{(\xi|\xi)}$. The inner product space is complete 
if any Cauchy sequence is convergent: Given 
a sequence of vectors $\{ \xi_n\}$ satisfying 
$\lim_{m,n \to \infty} \| \xi_m - \xi_n\| = 0$, 
we can find a vector $\xi$ such that 
$\lim_{n \to \infty} \| \xi_n - \xi \| = 0$. 
Any finite-dimensional inner product space is complete. 

When an inner product space is not complete, we can extend it 
to a complete one by adding ``virtual limits'', 
which is referred to as the completion of the inner product space.  

A Hilbert space is, by definition, a complete inner product space. 

The norm $\| A\|$ of a linear operator $A: \cH \to \cH$ 
on an inner product space is defined to be 
\[
\| A \| = \sup \{ \| A\xi \|/\| \xi\|; 0 \not= \xi \in \cH \}.
\]
A linear operator $A$ is said to be bounded if $\| A\| < \infty$. 
Any bounded operator is uniquely exetnded to a bounded linear 
operator of the completed Hilbert space. 

\begin{Example}
Let $\{ a_j\}$ be a sequence of complex numbers and 
$A: \C\N \to \C\N$ be a linear operator defined by 
$(A\xi)_j = a_j \xi_j$. Then 
\[
\| A\| = \sup \{ |a_j|; j \geq 0\}.
\]
\end{Example}

An eigenvector of a linear operator $A$ is a vector $\xi \not= 0$ 
satisfying $A\xi = \lambda \xi$ with $\lambda \in \C$. 
The complex number $\lambda$ is the eigenvalue of $\xi$. 
If $\lambda$ is an eigenvalue of $A$, the linear operator 
$\lambda I -A$ is not invertible. 
The converse implication holds when $\cH$ is finite-dimensional.

For a bounded linear operator $A$ on a Hilbert space, its spectrum 
$\sigma(A)$ is 
\[
\sigma(A) = \{ \lambda \in \C; \text{$\lambda I -A$ admits 
a bounded inverse} \}.
\]
When $\cH$ is finite-dimensional, 
$\sigma(A)$ is the set of eigenvalues of $A$. 

\begin{Example}
For the operator $A$ associated to a bounded sequence 
$\{ a_j\}$, 
\[
\sigma(A) = \text{closure of $\{ a_j; j \geq 0\}$}.
\]
\end{Example}

\begin{Exercise}
For the sequence $\{ a_j = e^{ij} \}$, 
$\sigma(A) = \T$. 
\end{Exercise}

\begin{Proposition}
The spectrum $\sigma(A)$ of a bounded linear operator $A$ 
is a non-empty closed subset of 
\[
\{ z \in \C; |z| \leq \| A\| \}.
\]
\end{Proposition}

Keywords and phrases:
finite graph, operator norm, spectral decomposition, Perron-Frobenius
eigenvector

Achievement:
To have a good comprehension of basic graphical notions.
To get enough skills in spectral features of matrices.
To get an acquaintance with what mathematical classifications mean.
To understand the fact that mathematical research would not be
accomplished as results of scheduled tasks.

Evaluation:
The course grade will be based entirely on homeworks
aasigned in the occasion of problem sessions.

Prerequisite:
Set theory (basic notions and terminology)
Linear algebra (undergraduate level)
Analysis (limit arguments in euclidean spaces)

Help: My offices hours are Wednesday 11:00-12:00 in C235 or
by appointment via e-mail (the email address can be found
at our group website http://www-mi.sci.ibaraki.ac.jp/)

References:
Any standard texts on linear algebra, calculus and set theory can be
used as your references. Other related documents will be supplied
in class.
